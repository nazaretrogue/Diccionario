\documentclass[11pt,a4paper]{article}
\usepackage[utf8]{inputenc}
\usepackage[spanish]{babel}	%Idioma
\usepackage{graphicx} 	%Añadir imágenes
\usepackage{geometry}	%Ajustar márgenes
\usepackage[export]{adjustbox}[2011/08/13]
\usepackage{float}
\usepackage{extarrows,pgffor}
\restylefloat{table}
\usepackage[hidelinks]{hyperref}
\usepackage{caption}
\usepackage[shortlabels]{enumitem}
\selectlanguage{spanish}

%Opciones de encabezado y pie de página:
\usepackage{fancyhdr}
\pagestyle{fancy}
\rhead{\empty}
\cfoot{\thepage}
\renewcommand{\headrulewidth}{0.4pt}
\renewcommand{\footrulewidth}{0.4pt}

%Opciones de fuente:
\usepackage[utf8]{inputenc}
\usepackage[default]{sourcesanspro}
\usepackage{sourcecodepro}
\usepackage[T1]{fontenc}

\setlength{\parindent}{15pt}
\setlength{\headheight}{15pt}
\setlength{\voffset}{10mm}

% Custom colors
\usepackage{color}
\definecolor{deepblue}{rgb}{0,0,0.5}
\definecolor{deepred}{rgb}{0.6,0,0}
\definecolor{deepgreen}{rgb}{0,0.5,0}

% Macros para definición de palabras
\newcommand{\palabra}[1]{\item \textbf{#1}}
\newcommand{\definicion}[1]{\begin{enumerate}
													\foreach \entry in {#1}
													{
														\item \entry
													}
												\end{enumerate}}
\newcommand{\sinonimo}[1]{\textcolor{deepgreen}{#1}}
\newcommand{\antonimo}[1]{\textcolor{deepred}{#1}}

\begin{document}
\begin{titlepage}

\begin{minipage}{\textwidth}

\centering
\includegraphics[width=0.5\textwidth]{./portada.png}\\

{\Huge\bfseries Diccionario personal\\}
\noindent\rule[-1ex]{\textwidth}{3pt}\\[3.5ex]
\end{minipage}
\end{titlepage}

\pagenumbering{gobble}
\pagenumbering{arabic}
\tableofcontents
\thispagestyle{empty}

\newpage

\section{Manual del diccionario personal}

Este diccionario contiene palabras que encuentro en libros o documentos que leo, así como palabra que escucho en películas y series. Muchas de estas palabras pueden ser palabras ``comunes'' de las cuales he podido olvidar el significado exacto al no usarlas de manera continuada, o bien pueden ser palabras que nunca he oído y que merecen la pena ser recordadas para poder expresar sentimientos, expresiones o gestos de una manera diferente a la habitual.\\

La estructura del diccionario es la siguiente:

\subsection{División por letra}

Como es normal, las palabras están divididas en secciones según la letra inicial. Dentro de cada sección, las palabras están ordenadas por orden alfabético.

\subsection{Uso de los colores}

Las palabras pueden tener sinónimos o antónimos y a veces es útil tener dichos términos a mano. A veces nos es más útil leer el antónimo de una palabra antes que usar la palabra en sí con una negación.\\

Por tanto, he añadido tanto sinónimos como antónimos a las palabras, haciéndolo visual para que encontrarlos sea fácil. Es fácil reconocer cual es cual:

\begin{itemize}
	\item Si la palabra es \sinonimo{verde}, es un \sinonimo{sinónimo}.
	\item Si la palabra es \antonimo{roja}, es un \antonimo{antónimo}.
\end{itemize}

\subsection{Aclaraciones en las palabras}

Las palabras pueden tener aclaraciones si cambian de significado según el contexto en el que se utilice.\\

En este caso, las aclaraciones se incluyen en mayúscula, entre corchetes o paréntesis cuadrados. Así, situar una palabra en su contexto es más sencillo.

\newpage

\section{Letra A}
\begin{itemize}
	\palabra{Abarloar}: Situar un barco de manera que uno de sus costados esté en contacto con el de otro buque o con un muelle, una batería, etc.
	\palabra{Abigarrado}:
		\definicion{Que tiene muchos colores mal combinados., Que está compuesto de muchos elementos muy diversos que no guardan orden o conexión entre ellos. (\sinonimo{Heterogéneo}).}
	\palabra{Abotargar}:
		\definicion{Embotar o entorpecer., [REFERIDO AL CUERPO] Hincharse\, generalmente por enfermedad.}
	\palabra{Acicate}:
		\definicion{Estímulo positivo que mueve a una persona a realizar una acción o a actuar de determinada manera., Espuela con solo una punta de hierro.}
	\palabra{Acuciar}:
		\definicion{Apremiar\, estimular o apurar a una persona para que haga algo., Algo que es un problema o una inquietud que requiere una rápida solución.}
	\palabra{Advenedizo}: Que se ha introducido en una posición, un ambiente o una actividad que no le corresponde por capacidad.
	\palabra{Agorero}: Que anuncia o predice males o desgracias.
	\palabra{Alegórico}: Que tiene significado simbólico.
	\palabra{Alicaído}: Que está triste, debilitado, desanimado o deprimido.
	\palabra{Aliteracion}: Figura retórica de dicción que consiste en la repetición de uno o varios sonidos dentro de una misma palabra o frase.
	\palabra{Alocución}: Discurso, generalmente breve, que pronuncia una autoridad o que dirige un jefe o superior a sus subordinados con ocasión de un acontecimiento especial.
	\palabra{Andanada}:
		\definicion{Conjunto de disparos que realizan los cañones o piezas de artillería de un barco al mismo tiempo., Conjunto de ataques o críticas negativas que se hacen contra alguien.}
	\palabra{Andrajoso}:
		\definicion{[REFERIDO A UNA PERSONA] Que viste con andrajos o de manera desaliñada. (\sinonimo{Harapiento, haraposo}), [REFERIDO A UNA PRENDA DE VESTIR] Que está roto\, sucio y muy gastado.].}
	\palabra{Anodadado}:
	\palabra{Anodino}:
	\palabra{Añagaza}:
	\palabra{Arduo}:
	\palabra{Arenga}:
	\palabra{Arrebato}:
	\palabra{Arrebujar}:
	\palabra{Arrellanarse}: Sentarse con comodidad, extendiendo y recostando el cuerpo.
	\palabra{Arrojado}: [FORMA DE SER]
	\palabra{Arteramente}: 
	\palabra{Astucia}:
	\palabra{Atemperado}:
	\palabra{Aterido}:
	\palabra{Atestado}:
	\palabra{Atisbo}:
	\palabra{Átona}: [REFERIDO A LA VOZ]
	\palabra{Aturdido}:
	\palabra{Austero}:
	\palabra{Avezado}:
	\palabra{Avieso}:
	\palabra{Avituallamiento}:
	\palabra{Azorado}:
\end{itemize}

\newpage

\section{Letra B}
\begin{itemize}
	\palabra{Beligerante}:
		\definicion{Que está en guerra con otro., Que está dispuesto a la hostilidad o que se muestra enfrentado o en desacuerdo a una persona\, un grupo o a una cosa.}
	\palabra{Bibisear}: Hablaren voz muy baja, casi susurrando.
	\palabra{Bucólico}: [GÉNERO DE POESÍA] Que trata de asuntos relacionados con la vida campestre y peripecias amorosas, suele tener a pastores como protagonistas y por lo común es dialogado.
	\palabra{Bufar}:
		\definicion{Resoplar con fuerza y furor., Manifestar enfado o malestar con sonidos semejantes a los bufidos de los animales\, generalmente acompañados de expresiones y gestos.}
\end{itemize}

\newpage

\section{Letra C}
\begin{itemize}
	\palabra{Cadalso}:
	\palabra{Candidez}:
	\palabra{Casquivana}:
		\definicion{Que es despreocupado e insensato y actúa sin ninguna formalidad., Que coquetea y establece relaciones de forma pasajera\, sin ningún compromiso serio.}
	\palabra{Cercenar}:
	\palabra{Cimbreante}:
	\palabra{Cincha}:
	\palabra{Cinismo}:
	\palabra{Condescendencia}:
	\palabra{Confraternizar}:
	\palabra{Conmoción}:
	\palabra{Consorte}: Referido a la monarquía, designa al cónyuge del monarca. Su tratamiento es diferente según su sexo y el reino al que pertenece.
	\palabra{Corcovear}:
	\palabra{Contrito}:
\end{itemize}

\newpage

\section{Letra D}
\begin{itemize}
	\palabra{Dádiva}: Cosa que se da como regalo.
	\palabra{Decoro}:
		\definicion{Comportamiento adecuado y respetuoso correspondiente a cada categoría y situación., Manera de comportarse con circunspección y gravedad.}
	\palabra{Dédalo}:
	\palabra{Deferencia}:
	\palabra{Delectación}:
	\palabra{Deliberado}:
	\palabra{Demasía}:
	\palabra{Demudado}:
	\palabra{Desharrapado}:
	\palabra{Deslavazado}:
		\definicion{Que es insustancial o insulso., Que carece de unión entre sus partes\, o está desordenado o mal compuesto.}
	\palabra{Despectivo}:
	\palabra{Desquiciar}:
	\palabra{Desvencijado}:
	\palabra{Diáfano}:
	\palabra{Dicción}:
	\palabra{Diegético}:
	\palabra{Dilapidar}:
	\palabra{Disidente}:
		\definicion{Que se separa del partido\, la religión\, el gobierno o el colectivo ideológico al que pertenece\, por no estar de acuerdo con su doctrina\, creencia\, sistema\, etc., Que es propio y característico de este tipo de personas.}
	\palabra{Disoluto}:
	\palabra{Docto}:
	\palabra{Ducho}:
\end{itemize}

\newpage

\section{Letra E}
\begin{itemize}
	\palabra{Ecléctico}:
	\palabra{Eludir}: Evitar una dificultad, obligación, etc., con algún artificio o estratagema.
	\palabra{Encalar}:
	\palabra{Entusiasmo}:
	\palabra{Envarar}:
		\definicion{Impedir o entorpecer el movimiento de un miembro del cuerpo., Volverse o mostrarse soberbio.}
	\palabra{Envite}:	
	\palabra{Escarnio}:
	\palabra{Escudarse}:
	\palabra{Esoterismo}:
		\definicion{Cualidad de lo que está oculto a los sentidos y a la ciencia o es difícil de entender., Conjunto de conocimientos y prácticas relacionados con la magia\, la alquimia\, la astrología y materias semejantes\, que no se basan en la experimentación científica.}
	\palabra{Espaldarazo}:
		\definicion{Ayuda o empuje que recibe una persona o una cosa en su trayectoria hacia un determinado fin social o profesional., Reconocimiento de los méritos o habilidades de una persona en su profesión o en la actividad que realiza.}
	\palabra{Espanto}:
	\palabra{Espartano}:
	\palabra{Esquife}:
	\palabra{Estertor}:
	\palabra{Exacerbar}:
	\palabra{Exasperación}:
	\palabra{Exhortar}:
	\palabra{Expeditivo}:
\end{itemize}

\newpage

\section{Letra F}
\begin{itemize}
	\palabra{Fatiga}:
		\definicion{Cansancio o hastío., Molestia ocasionada por un esfuerzo más o menos prolongado o por otras causas y que en ocasiones produce alteraciones físicas., [EXPRESIÓN] ``Darle fatiga algo a alguien'': sentir reparos o miramientos.}
	\palabra{Fehaciente}: Que prueba o da fe de algo de forma indudable. (\sinonimo{Fidedigno, indudable}).
	\palabra{Forastero}: Que es o ha venido de otro lugar.
	\palabra{Frívolo}: Que no concede a las cosas la importancia que merecen, no las hace con la seriedad, el sentimiento o el interés requeridos y solo piensa en el aspecto divertido o lúdico de la vida. (\sinonimo{Veleidoso, fútil, vano, intrascendente}).
	\palabra{Fútil}: Que carece de importancia o interés por su falta de fundamento. (\sinonimo{Frívolo, insustancial}).
\end{itemize}

\newpage

\section{Letra G}
\begin{itemize}
	\palabra{Gallardía}: Que se hace u ocurre de manera oculta. (\sinonimo{Encubierto, furtivo, oculto}).
\end{itemize}

\newpage

\section{Letra H}
\begin{itemize}
	\palabra{Hampón}: Persona que vive de forma marginal cometiendo acciones delictivas de manera habitual. (\sinonimo{Maleante}).
	\palabra{Harapiento}:
	\palabra{Hastiado}:
	\palabra{Hedonismo}:
	\palabra{Henchido}
	\palabra{Hirsuto}:
	\palabra{Horadar}:
	\palabra{Huraño}:
\end{itemize}

\newpage

\section{Letra I}
\begin{itemize}
	\palabra{Impasible}:
	\palabra{Impávido}:
	\palabra{Impertérrito}:
	\palabra{Impertinente}:
	\palabra{Imprecar}: Expresar vivamente el deseo de que alguien sufra un daño o un mal.
	\palabra{Indolente}:
	\palabra{Indomable}:
	\palabra{Infamia}:
	\palabra{Infastuo}:
	\palabra{Inferencia}:
	\palabra{Inherente}:
	\palabra{Imperioso}:
	\palabra{Inquietante}:
	\palabra{Inserto}:
	\palabra{Insidioso}:
	\palabra{Insulso}:
	\palabra{Intempestivo}:
	\palabra{Intendencia}:
		\definicion{Control y administración de un servicio o del abastecimiento de una colectividad., Cuerpo del ejército encargado de proporcionar y organizar todo lo que necesitan las fuerzas armadas o campamentos para funcionar de forma adecuada.}
	\palabra{Intrépido}:
\end{itemize}

\newpage

\section{Letra L}
\begin{itemize}
	\palabra{Lasca}: Fragmento plano y delgado desprendido de una piedra.
	\palabra{Locuaz}: Que habla mucho; en especial referido a cuando lo hace con soltura o facilidad. (\sinonimo{Hablador, verboso, parlanchín}).
\end{itemize}

\newpage

\section{Letra M}
\begin{itemize}
	\palabra{Macilento}: Que está flaco y demacrado o tiene la cara flaca y pálida. (\sinonimo{Flaco, demacrado, mustio}).
	\palabra{Majestuoso}: Que es capaz de infundir admiración y respeto por su solemnidad, elegancia o grandeza. (\sinonimo{Mayestático, augusto, solemne, imponente}).
	\palabra{Marisma}: Terreno bajo y pantanoso que inundan las aguas del mar.
	\palabra{Megalomanía}:
		\definicion{Trastorno mental que padece la persona que se cree socialmente muy importante poseedora de enormes riquezas y capaz de hacer grandes cosas., Actitud que tiene la persona que se comporta como si tuviera una posición social y económica muy superiores a las reales.}
	\palabra{Mezcolanza}: [DESPECTIVO] Mezcla extraña, a veces confusa e incluso ridícula de personas, cosas o ideas opuestas o inconexas.
	\palabra{Misiva}: Carta que se envía a una persona para informarle de algo.
	\palabra{Mordaz}: Que es crítico, tiene ironía aguda y malintencionada. (\sinonimo{Incisivo, satírico}).
	\palabra{Mordiente}:
		\definicion{Sustancia química que sirve para fijar el color o el pan (lámina muy fina de oro\, plata\, etc.) a una cosa., Ácido con que se desgasta una plancha para grabarla.}
	\palabra{Moscoso}:
	\palabra{Mundano}:
\end{itemize}

\newpage

\section{Letra N}
\begin{itemize}
	\palabra{Necio}: Que insiste en los propios errores o se aferra a ideas o posturas equivocadas, demostrando con ello poca inteligencia.
\end{itemize}

\newpage

\section{Letra O}
\begin{itemize}
	\palabra{Ominoso}: Que es abominable y merece ser condenado y aborrecido.
	\palabra{Ordalía}: Prueba a la que eran sometidos los acusados en la Edad Media para averiguar su culpabilidad o inocencia; como las del duelo, el fuego, el hierro candente, etc.
	\palabra{Oriundo}:
	\palabra{Ostensión}: Manifestación de algo.
	\palabra{Ostentación}:
		\definicion{Exhibición que se hace de una cosa con vanidad o presunción., Manifestación excesiva de lujo o riqueza.}
	\palabra{Ostracismo}:
		\definicion{En la Grecia antigua destierro a que se condenaba a los ciudadanos que se consideraban sospechosos o peligrosos para la ciudad., Aislamiento voluntario o forzoso de la vida pública que sufre una persona generalmente motivado por cuestiones políticas.}
	\palabra{Otrora}: En un tiempo pasado indeterminado que queda lejano del momento del cual se habla; generalmente contrapone la situación descrita con otra situación anterior.
\end{itemize}

\newpage

\section{Letra P}
\begin{itemize}
	\palabra{Pasquín}:
	\palabra{Pecado venial}: Negligencia, tropiezo o vacilación en el seguimiento de Cristo. (\sinonimo{Pecado leve}).
	\palabra{Petulante}:
	\palabra{Piedad}:
	\palabra{Portentoso}:
	\palabra{Prerrogativa}:
	\palabra{Prístino}:
	\palabra{Promontorio}:
	\palabra{Pugnar}:
\end{itemize}

\newpage

\section{Letra Q}
\begin{itemize}
	\palabra{Queda}: [REFERIDO A LA VOZ] Baja o que apenas se oye.
	\palabra{Quemar(se a lo bonzo)}: [EXPRESIÓN] Forma de autoinmolación proveniente de los monjes budistas japoneses de los años 60.
\end{itemize}

\newpage

\section{Letra R}
\begin{itemize}
	\palabra{Regio}:
	\palabra{Reservado}: [REFERIDO A UNA PERSONA]
	\palabra{Resquemor}:
	\palabra{Reticente}:
		\definicion{Hecho de insinuar o no decir directamente algo\, generalmente con intención maliciosa., Desconfianza o cautela que inspiran ciertas personas\, actos o dichos.}
	\palabra{Rezongar}:
	\palabra{Rielar}:
	\palabra{Rigor}:
\end{itemize}

\newpage

\section{Letra S}
\begin{itemize}
	\palabra{Sensatez}:
	\palabra{Servil}:
	\palabra{Sibilino}:
		\definicion{Que es propio o característico de la sibila., Que es misterioso porque parece que encierra un secreto importante o que puede tener varios significados ocultos.}
	\palabra{Silo}: Construcción diseñada para almacenar grano y otros materiales a granel; son parte del ciclo de acopio de la agricultura.
	\palabra{Sináptico}:
	\palabra{Sobrio}:
	\palabra{Solaz}:
	\palabra{Solícito}:
	\palabra{Sombrío}:
	\palabra{Subrepticio}:
	\palabra{Subterfugio}:
	\palabra{Suburbios}:
	\palabra{Superchería}: Engaño o fraude consistente en sustituir una cosa verdadera por una falsa.
	\palabra{Superfluo}:
	\palabra{Sutil}:
	\palabra{Sutileza}:
\end{itemize}

\newpage

\section{Letra T}
\begin{itemize}
	\palabra{Tenaz}: Que pone mucho empeño y no desiste en algo que quiere hacer o conseguir.
	\palabra{Tesón}: Firmeza, decisión y perseverancia que una persona pone en la realización de una cosa. (\sinonimo{Empeño, constancia, perseverancia}).
	\palabra{Tocido}: [REFERIDO AL GESTO] Que muestra enfado o disgusto.
	\palabra{Torvo}:
		\definicion{Que tiene aspecto fiero y airado., Que es propio de estas personas.}
	\palabra{Tozudo}: Que se mantiene firme o inamovible en su actitud, aunque se le den razones en contra.
	\palabra{Trocha}: Camino estrecho, especialmente el que sirve de atajo. (\sinonimo{Vereda, sendero}).
\end{itemize}

\newpage

\section{Letra V}
\begin{itemize}
	\palabra{Vacilar}:
		\definicion{Moverse de un lado a otro por falta de estabilidad o equilibrio., Estar indeciso o titubeante\, sin decidirse a hacer o decir algo o sin escoger entre varias cosas.}
	\palabra{Vacuo}:
	\palabra{Vaharada}:
		\definicion{Acción de echar aliento o vaho., Ráfaga de vaho\, olor\, calor\, etc.}
	\palabra{Vanidad}: Orgullo de la persona que tiene en un alto concepto sus propios méritos y un afán excesivo de ser admirado y considerado por ellos. (\sinonimo{Presunción, engreimiento, jactancia, fatuidad, soberbia, orgullo, ostentación}).
	\palabra{Vehemente}:
		\definicion{Que se manifiesta con ímpetu\, viveza o pasión., [REFERIDO A UNA PERSONA] Que obra de forma irreflexiva y apasionada\, dejándose llevar por los sentimientos o los impulsos.}
	\palabra{Vilipendio}: Desprecio o denigración grave.
	\palabra{Vindicar}:
		\definicion{Defender o exculpar a una persona injustamente atacada\, especialmente por escrito., Vengar.}
	\palabra{Violento}:
		\definicion{[REFERIDO A UNA SITUACIÓN] Embarazosa., Que está fuera de su natural estado\, situación o modo.}
	\palabra{Vivaquear}: Acampar de noche al aire libre.
	\palabra{Vorágine}:
\end{itemize}

\end{document}