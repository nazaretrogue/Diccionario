\documentclass[11pt,a4paper]{article}
\usepackage[utf8]{inputenc}
\usepackage[spanish]{babel}	%Idioma
\usepackage{graphicx} 	%Añadir imágenes
\usepackage{geometry}	%Ajustar márgenes
\usepackage[export]{adjustbox}[2011/08/13]
\usepackage{float}
\usepackage{extarrows,pgffor}
\restylefloat{table}
\usepackage[hidelinks]{hyperref}
\usepackage{caption}
\usepackage[shortlabels]{enumitem}
\selectlanguage{spanish}

%Opciones de encabezado y pie de página:
\usepackage{fancyhdr}
\pagestyle{fancy}
\rhead{\empty}
\cfoot{\thepage}
\renewcommand{\headrulewidth}{0.4pt}
\renewcommand{\footrulewidth}{0.4pt}

%Opciones de fuente:
\usepackage[utf8]{inputenc}
\usepackage[default]{sourcesanspro}
\usepackage{sourcecodepro}
\usepackage[T1]{fontenc}

\setlength{\parindent}{15pt}
\setlength{\headheight}{15pt}
\setlength{\voffset}{10mm}

% Custom colors
\usepackage{color}
\definecolor{deepblue}{rgb}{0,0,0.5}
\definecolor{deepred}{rgb}{0.6,0,0}
\definecolor{deepgreen}{rgb}{0,0.5,0}

% Macros para definición de palabras
\newcommand{\palabra}[1]{\item \textbf{#1}}
\newcommand{\definicion}[1]{\begin{enumerate}
													\foreach \entry in {#1}
													{
														\item \entry
													}
												\end{enumerate}}

\begin{document}
\begin{titlepage}

\begin{minipage}{\textwidth}

\centering
\includegraphics[width=0.5\textwidth]{./portada.png}\\

{\Huge\bfseries Diccionario personal\\}
\noindent\rule[-1ex]{\textwidth}{3pt}\\[3.5ex]
\end{minipage}
\end{titlepage}

\pagenumbering{gobble}
\pagenumbering{arabic}
\tableofcontents
\thispagestyle{empty}

\newpage

\section{Letra A}
\begin{itemize}
	\palabra{Acicate}:
		\definicion{Estímulo positivo que mueve a una persona a realizar una acción o a actuar de determinada manera., Espuela con solo una punta de hierro.}
	\palabra{Acuciar}:
		\definicion{Apremiar\, estimular o apurar a una persona para que haga algo., Algo que es un problema o una inquietud que requiere una rápida solución.}
	\palabra{Aliteracion}: Figura retórica de dicción que consiste en la repetición de uno o varios sonidos dentro de una misma palabra o frase.
	\palabra{Andanada}:
		\definicion{Conjunto de disparos que realizan los cañones o piezas de artillería de un barco al mismo tiempo., Conjunto de ataques o críticas negativas que se hacen contra alguien.}
	\palabra{Arrellanarse}: Sentarse con comodidad, extendiendo y recostando el cuerpo.
\end{itemize}

\newpage

\section{Letra B}
\begin{itemize}
	\palabra{Beligerante}:
		\definicion{Que está en guerra con otro., Que está dispuesto a la hostilidad o que se muestra enfrentado o en desacuerdo a una persona\, un grupo o a una cosa.}
\end{itemize}

\newpage

\section{Letra C}
\begin{itemize}
	\palabra{Casquivana}:
		\definicion{Que es despreocupado e insensato y actúa sin ninguna formalidad., Que coquetea y establece relaciones de forma pasajera\, sin ningún compromiso serio.}
	\palabra{Consorte}: Referido a la monarquía, designa al cónyuge del monarca. Su tratamiento es diferente según su sexo y el reino al que pertenece.
\end{itemize}

\newpage

\section{Letra D}
\begin{itemize}
	\palabra{Dádiva}: Cosa que se da como regalo.
	\palabra{Decoro}:
		\definicion{Comportamiento adecuado y respetuoso correspondiente a cada categoría y situación., Manera de comportarse con circunspección y gravedad.}
	\palabra{Deslavazado}:
		\definicion{Que es insustancial o insulso., Que carece de unión entre sus partes\, o está desordenado o mal compuesto.}
	\palabra{Disidente}:
		\definicion{Que se separa del partido\, la religión\, el gobierno o el colectivo ideológico al que pertenece\, por no estar de acuerdo con su doctrina\, creencia\, sistema\, etc., Que es propio y característico de este tipo de personas.}
\end{itemize}

\newpage

\section{Letra E}
\begin{itemize}
	\palabra{Eludir}: Evitar una dificultad, obligación, etc., con algún artificio o estratagema.
	\palabra{Envarar}:
		\definicion{Impedir o entorpecer el movimiento de un miembro del cuerpo., Volverse o mostrarse soberbio.}
	\palabra{Esoterismo}:
		\definicion{Cualidad de lo que está oculto a los sentidos y a la ciencia o es difícil de entender., Conjunto de conocimientos y prácticas relacionados con la magia\, la alquimia\, la astrología y materias semejantes\, que no se basan en la experimentación científica.}
	\palabra{Espaldarazo}:
		\definicion{Ayuda o empuje que recibe una persona o una cosa en su trayectoria hacia un determinado fin social o profesional., Reconocimiento de los méritos o habilidades de una persona en su profesión o en la actividad que realiza.}
\end{itemize}

\newpage

\section{Letra I}
\begin{itemize}
	\palabra{Imprecar}: Expresar vivamente el deseo de que alguien sufra un daño o un mal.
	\palabra{Intendencia}:
		\definicion{Control y administración de un servicio o del abastecimiento de una colectividad., Cuerpo del ejército encargado de proporcionar y organizar todo lo que necesitan las fuerzas armadas o campamentos para funcionar de forma adecuada.}
\end{itemize}

\newpage

\section{Letra M}
\begin{itemize}
	\palabra{Mezcolanza}: [DESPECTIVO] Mezcla extraña, a veces confusa e incluso ridícula de personas, cosas o ideas opuestas o inconexas.
	\palabra{Misiva}: Carta que se envía a una persona para informarle de algo.
	\palabra{Mordiente}:
		\definicion{Sustancia química que sirve para fijar el color o el pan (lámina muy fina de oro\, plata\, etc.) a una cosa., Ácido con que se desgasta una plancha para grabarla.}
\end{itemize}

\newpage

\section{Letra O}
\begin{itemize}
	\palabra{Ominoso}: Que es abominable y merece ser condenado y aborrecido.
	\palabra{Ordalía}: Prueba a la que eran sometidos los acusados en la Edad Media para averiguar su culpabilidad o inocencia; como las del duelo\, el fuego\, el hierro candente\, etc.
\end{itemize}

\section{Letra R}
\begin{itemize}
	\palabra{Reticente}:
		\definicion{Hecho de insinuar o no decir directamente algo\, generalmente con intención maliciosa., Desconfianza o cautela que inspiran ciertas personas\, actos o dichos.}
\end{itemize}

\newpage

\section{Letra Q}
\begin{itemize}
	\palabra{Queda}: Referido a la voz, baja o que apenas se oye.
\end{itemize}

\newpage

\section{Letra S}
\begin{itemize}
	\palabra{Sibilino}:
		\definicion{Que es propio o característico de la sibila., Que es misterioso porque parece que encierra un secreto importante o que puede tener varios significados ocultos.}
	\palabra{Silo}: Construcción diseñada para almacenar grano y otros materiales a granel; son parte del ciclo de acopio de la agricultura.
	\palabra{Superchería}: Engaño o fraude consistente en sustituir una cosa verdadera por una falsa.
\end{itemize}

\newpage

\section{Letra T}
\begin{itemize}
	\palabra{Torvo}:
		\definicion{Que tiene aspecto fiero y airado., Que es propio de estas personas.}
\end{itemize}

\newpage

\section{Letra V}
\begin{itemize}
	\palabra{Vilipendio}: Desprecio o denigración grave.
\end{itemize}

\end{document}